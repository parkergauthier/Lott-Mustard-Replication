\documentclass{article}

% these packages let you do math
\usepackage{amsmath}
\usepackage{amssymb}

% we need these packages for fancy R tables
\usepackage{booktabs}
\usepackage{float}
\usepackage{colortbl}
\usepackage{xcolor}

% these packages play with the spacing/margins of the document. Uncomment the commands on lines 16 and 17 to see what they do.
\usepackage{a4wide}
\usepackage{setspace}
\usepackage{geometry}
\usepackage{parskip}
%\doublespacing
%\geometry{margin=1.5in}

% set the author, title, and date of the document. \maketitle adds it to the document.
\author{Parker Gauthier}
\title{Lott & Mustard Replication}
\date{Sping 2022}

\begin{document}
\maketitle

\section{The First Section}

Below is a brief analysis of incarceration status by race in the year 2002.  Data is collected from the NLSY97 website.

To set up this analysis, I began by checking to see if an indiviual was incarcerated for at least one month in 2002.  I then counted how many indiviuals were incarcerated, grouped by race and gender.

\section{Plot 1}

The plot above gives us a look into how many people were incarcerated in 2002 by their race and gender.

We can see that the males made up a vast majority of the incarcerated population in 2002.  Furthermore, black americans were incarcerated more than any other race.

\section{Tables}

The table below helps us see, more specifically, the total number of incarcerated people by race.


% Table created by stargazer v.5.2.3 by Marek Hlavac, Social Policy Institute. E-mail: marek.hlavac at gmail.com
% Date and time: Tue, May 03, 2022 - 1:06:22 AM
\begin{table}[!htbp] \centering 
  \caption{} 
  \label{} 
\begin{tabular}{@{\extracolsep{5pt}} cccc} 
\\[-1.8ex]\hline 
\hline \\[-1.8ex] 
Target Variable & TWFE & Calloway\_SantAnna & CallowaySantAnna \\ 
\hline \\[-1.8ex] 
lvio & -0.0572 & $$-$0.010$ & -0.01 \\ 
 & (0.0234) & $0.027$ & (0.0268) \\ 
lmur & 0.0085 & $0.013$ & 0.0129 \\ 
 & (0.0137) & $0.012$ & (0.012) \\ 
laga & -0.0504 & $$-$0.049$ & -0.0486 \\ 
 & (0.0396) & $0.026$ & (0.0258) \\ 
lbur & -0.0494 & $0.020$ & 0.02 \\ 
 & (0.028) & $0.029$ & (0.0286) \\ 
laut & -0.0509 & $0.005$ & 0.0048 \\ 
 & (0.0305) & $0.044$ & (0.0439) \\ 
lpro & -0.0178 & $0.039$ & 0.0388 \\ 
 & (0.0313) & $0.036$ & (0.0365) \\ 
lrap & -0.0243 & $$-$0.016$ & -0.0158 \\ 
 & (0.0194) & $0.015$ & (0.0151) \\ 
lrob & 0.014 & $0.030$ & 0.0303 \\ 
 & (0.013) & $0.016$ & (0.0162) \\ 
llar & 0.0345 & $0.011$ & 0.0107 \\ 
 & (0.0297) & $0.043$ & (0.0429) \\ 
\hline \\[-1.8ex] 
\end{tabular} 
\end{table} 


To help further our analysis, a regression can be run to see how one's race impacts their chances of being incarcerated.

Nested in our constant are the variables "Black" and "female," so we can see that, when using this as our baseline, other races have slightly diminished odds of being incarcerated.  Additionally, the being a male significantly increases one's odds of being incarcerated.

It is important to note that these variables do not clearly indicate that one will have greater chances of being incarcerated based on the demographic traits laid out above.  Rather, it reflects the current chance that, given a random individual in this dataset, that they will have been incarcerated in 2002.


\end{document}