% Options for packages loaded elsewhere
\PassOptionsToPackage{unicode}{hyperref}
\PassOptionsToPackage{hyphens}{url}
%
\documentclass[
]{article}
\usepackage{lmodern}
\usepackage{amssymb,amsmath}
\usepackage{ifxetex,ifluatex}
\ifnum 0\ifxetex 1\fi\ifluatex 1\fi=0 % if pdftex
  \usepackage[T1]{fontenc}
  \usepackage[utf8]{inputenc}
  \usepackage{textcomp} % provide euro and other symbols
\else % if luatex or xetex
  \usepackage{unicode-math}
  \defaultfontfeatures{Scale=MatchLowercase}
  \defaultfontfeatures[\rmfamily]{Ligatures=TeX,Scale=1}
\fi
% Use upquote if available, for straight quotes in verbatim environments
\IfFileExists{upquote.sty}{\usepackage{upquote}}{}
\IfFileExists{microtype.sty}{% use microtype if available
  \usepackage[]{microtype}
  \UseMicrotypeSet[protrusion]{basicmath} % disable protrusion for tt fonts
}{}
\makeatletter
\@ifundefined{KOMAClassName}{% if non-KOMA class
  \IfFileExists{parskip.sty}{%
    \usepackage{parskip}
  }{% else
    \setlength{\parindent}{0pt}
    \setlength{\parskip}{6pt plus 2pt minus 1pt}}
}{% if KOMA class
  \KOMAoptions{parskip=half}}
\makeatother
\usepackage{xcolor}
\IfFileExists{xurl.sty}{\usepackage{xurl}}{} % add URL line breaks if available
\IfFileExists{bookmark.sty}{\usepackage{bookmark}}{\usepackage{hyperref}}
\hypersetup{
  pdftitle={Untitled},
  pdfauthor={Parker Gauthier},
  hidelinks,
  pdfcreator={LaTeX via pandoc}}
\urlstyle{same} % disable monospaced font for URLs
\usepackage[margin=1in]{geometry}
\usepackage{color}
\usepackage{fancyvrb}
\newcommand{\VerbBar}{|}
\newcommand{\VERB}{\Verb[commandchars=\\\{\}]}
\DefineVerbatimEnvironment{Highlighting}{Verbatim}{commandchars=\\\{\}}
% Add ',fontsize=\small' for more characters per line
\usepackage{framed}
\definecolor{shadecolor}{RGB}{248,248,248}
\newenvironment{Shaded}{\begin{snugshade}}{\end{snugshade}}
\newcommand{\AlertTok}[1]{\textcolor[rgb]{0.94,0.16,0.16}{#1}}
\newcommand{\AnnotationTok}[1]{\textcolor[rgb]{0.56,0.35,0.01}{\textbf{\textit{#1}}}}
\newcommand{\AttributeTok}[1]{\textcolor[rgb]{0.77,0.63,0.00}{#1}}
\newcommand{\BaseNTok}[1]{\textcolor[rgb]{0.00,0.00,0.81}{#1}}
\newcommand{\BuiltInTok}[1]{#1}
\newcommand{\CharTok}[1]{\textcolor[rgb]{0.31,0.60,0.02}{#1}}
\newcommand{\CommentTok}[1]{\textcolor[rgb]{0.56,0.35,0.01}{\textit{#1}}}
\newcommand{\CommentVarTok}[1]{\textcolor[rgb]{0.56,0.35,0.01}{\textbf{\textit{#1}}}}
\newcommand{\ConstantTok}[1]{\textcolor[rgb]{0.00,0.00,0.00}{#1}}
\newcommand{\ControlFlowTok}[1]{\textcolor[rgb]{0.13,0.29,0.53}{\textbf{#1}}}
\newcommand{\DataTypeTok}[1]{\textcolor[rgb]{0.13,0.29,0.53}{#1}}
\newcommand{\DecValTok}[1]{\textcolor[rgb]{0.00,0.00,0.81}{#1}}
\newcommand{\DocumentationTok}[1]{\textcolor[rgb]{0.56,0.35,0.01}{\textbf{\textit{#1}}}}
\newcommand{\ErrorTok}[1]{\textcolor[rgb]{0.64,0.00,0.00}{\textbf{#1}}}
\newcommand{\ExtensionTok}[1]{#1}
\newcommand{\FloatTok}[1]{\textcolor[rgb]{0.00,0.00,0.81}{#1}}
\newcommand{\FunctionTok}[1]{\textcolor[rgb]{0.00,0.00,0.00}{#1}}
\newcommand{\ImportTok}[1]{#1}
\newcommand{\InformationTok}[1]{\textcolor[rgb]{0.56,0.35,0.01}{\textbf{\textit{#1}}}}
\newcommand{\KeywordTok}[1]{\textcolor[rgb]{0.13,0.29,0.53}{\textbf{#1}}}
\newcommand{\NormalTok}[1]{#1}
\newcommand{\OperatorTok}[1]{\textcolor[rgb]{0.81,0.36,0.00}{\textbf{#1}}}
\newcommand{\OtherTok}[1]{\textcolor[rgb]{0.56,0.35,0.01}{#1}}
\newcommand{\PreprocessorTok}[1]{\textcolor[rgb]{0.56,0.35,0.01}{\textit{#1}}}
\newcommand{\RegionMarkerTok}[1]{#1}
\newcommand{\SpecialCharTok}[1]{\textcolor[rgb]{0.00,0.00,0.00}{#1}}
\newcommand{\SpecialStringTok}[1]{\textcolor[rgb]{0.31,0.60,0.02}{#1}}
\newcommand{\StringTok}[1]{\textcolor[rgb]{0.31,0.60,0.02}{#1}}
\newcommand{\VariableTok}[1]{\textcolor[rgb]{0.00,0.00,0.00}{#1}}
\newcommand{\VerbatimStringTok}[1]{\textcolor[rgb]{0.31,0.60,0.02}{#1}}
\newcommand{\WarningTok}[1]{\textcolor[rgb]{0.56,0.35,0.01}{\textbf{\textit{#1}}}}
\usepackage{graphicx,grffile}
\makeatletter
\def\maxwidth{\ifdim\Gin@nat@width>\linewidth\linewidth\else\Gin@nat@width\fi}
\def\maxheight{\ifdim\Gin@nat@height>\textheight\textheight\else\Gin@nat@height\fi}
\makeatother
% Scale images if necessary, so that they will not overflow the page
% margins by default, and it is still possible to overwrite the defaults
% using explicit options in \includegraphics[width, height, ...]{}
\setkeys{Gin}{width=\maxwidth,height=\maxheight,keepaspectratio}
% Set default figure placement to htbp
\makeatletter
\def\fps@figure{htbp}
\makeatother
\setlength{\emergencystretch}{3em} % prevent overfull lines
\providecommand{\tightlist}{%
  \setlength{\itemsep}{0pt}\setlength{\parskip}{0pt}}
\setcounter{secnumdepth}{-\maxdimen} % remove section numbering
\usepackage{booktabs}
\usepackage{longtable}
\usepackage{array}
\usepackage{multirow}
\usepackage{wrapfig}
\usepackage{float}
\usepackage{colortbl}
\usepackage{pdflscape}
\usepackage{tabu}
\usepackage{threeparttable}
\usepackage{threeparttablex}
\usepackage[normalem]{ulem}
\usepackage{makecell}
\usepackage{xcolor}

\title{Untitled}
\author{Parker Gauthier}
\date{4/15/2022}

\begin{document}
\maketitle

\hypertarget{lott-and-mustard-replication-excercise}{%
\section{Lott and Mustard Replication
Excercise}\label{lott-and-mustard-replication-excercise}}

\hypertarget{introduction}{%
\subsection{Introduction}\label{introduction}}

For years the relationship between crime and gun laws has been a topic
of significant contention in the United States. Some argue that
restricting gun ownership will deter gun violence, while those on the
other end of the aisle believe in quite the opposite. Researchers John
Lott and David Mustard aimed to clear up this argument in their paper,
``Crime, Deterrence, and Right-to-Carry Concealed Handguns.'' The
authors attempt to tackle this problem by analyzing the effects of
concealed carry laws on various crime rates using econometric models
aimed at inferring causality. The authors conclude that when states give
their citizens the right to carry a concealed firearm, violent crime
rates decline without a significant increase in accidental gun deaths.
Their findings are quite intriguing, but were their methods sound?

\begin{verbatim}
\end{verbatim}

The goal of the analysis below will be to assess the models used by Lott
and Mustard and see how they stack up to contemporary causal inference
methods. We will look at the same data used by the researchers and first
attempt to replicate their results. We will then used other predictive
models to see if we see the same effects depicted by the researchers.
Ultimately, we will assess what methods are the most effective in
determining causal effects and highlight the implications of using a
faulty model.

\hypertarget{background-and-economic-theory}{%
\subsection{Background and Economic
Theory}\label{background-and-economic-theory}}

\begin{table}

\caption{\label{tab:unnamed-chunk-3}Table 1}
\centering
\begin{tabular}[t]{l|l}
\hline
State Name & Year Treated\\
\hline
Alabama & Treated Entire Period\\
\hline
Connecticut & Treated Entire Period\\
\hline
New Hampshire & Treated Entire Period\\
\hline
North Dakota & Treated Entire Period\\
\hline
South Dakota & Treated Entire Period\\
\hline
Vermont & Treated Entire Period\\
\hline
Washington & Treated Entire Period\\
\hline
Indiana & 1981\\
\hline
Maine & 1986\\
\hline
Florida & 1988\\
\hline
Virginia & 1989\\
\hline
Georgia & 1990\\
\hline
Pennsylvania & 1990\\
\hline
West Virginia & 1990\\
\hline
Idaho & 1991\\
\hline
Mississippi & 1991\\
\hline
Oregon & 1991\\
\hline
Montana & 1992\\
\hline
\end{tabular}
\end{table}

\hypertarget{data}{%
\subsection{Data}\label{data}}

\begin{table}

\caption{\label{tab:unnamed-chunk-4}Summary Statistics}
\centering
\begin{tabular}[t]{l|r|r}
\hline
  & Mean & Sd\\
\hline
Arest Rates - Violent Crime & 41.09 & 22.20\\
\hline
Property Crime & 16.92 & 4.68\\
\hline
Murder & 91.30 & 55.94\\
\hline
Rape & 41.02 & 17.39\\
\hline
Robbery & 31.46 & 13.59\\
\hline
Burglary & 13.80 & 4.57\\
\hline
Larceny & 18.54 & 5.20\\
\hline
Auto Theft & 22.35 & 37.61\\
\hline
Crime Rates - Violent Crime & 483.93 & 318.94\\
\hline
Property Crime & 4618.34 & 1210.46\\
\hline
Murder & 7.77 & 6.88\\
\hline
Rape & 33.98 & 15.07\\
\hline
Agravated Assault & 278.76 & 159.65\\
\hline
Robbery & 163.42 & 176.25\\
\hline
Auto Theft & 410.30 & 231.15\\
\hline
Burglary & 1239.34 & 417.76\\
\hline
Larceny & 2968.71 & 751.02\\
\hline
\end{tabular}
\end{table}

\hypertarget{empircal-model-and-estimation}{%
\subsection{Empircal Model and
Estimation}\label{empircal-model-and-estimation}}

The first model we will look at will be similar to the model originally
used by Lott \& Mustard in their paper. This model, `Twoway Fixed
Effects,' is a type of difference-in-difference design where we compare
our observations to a fixed effect to identify whether a treated group
has a different trend than a control group.

\begin{Shaded}
\begin{Highlighting}[]
\NormalTok{effects =}\StringTok{ }\KeywordTok{foreach}\NormalTok{(}\DataTypeTok{y =} \DecValTok{1}\OperatorTok{:}\NormalTok{len_, }\DataTypeTok{.combine =}\NormalTok{ rbind) }\OperatorTok\StringTok{ }\NormalTok{\{}
\NormalTok{  y_var =}\StringTok{ }\NormalTok{y_vars[y]}
  
\NormalTok{  twfe_name =}\StringTok{ }\KeywordTok{paste}\NormalTok{(}\StringTok{"twfe"}\NormalTok{, y_var, }\DataTypeTok{sep =} \StringTok{"_"}\NormalTok{)}
\NormalTok{  fname_twfe =}\StringTok{ }\KeywordTok{paste0}\NormalTok{(twfe_name, }\StringTok{".RDs"}\NormalTok{)}
  
\NormalTok{  bacon_name =}\StringTok{ }\KeywordTok{paste}\NormalTok{(}\StringTok{"twfe"}\NormalTok{, y_var, }\DataTypeTok{sep=}\StringTok{"_"}\NormalTok{)}
\NormalTok{  fname_bacon =}\StringTok{ }\KeywordTok{paste0}\NormalTok{(bacon_name, }\StringTok{".RDs"}\NormalTok{)}
  
\NormalTok{  cs_name =}\StringTok{ }\KeywordTok{paste}\NormalTok{(}\StringTok{"CS"}\NormalTok{, y_var, }\DataTypeTok{sep =} \StringTok{"_"}\NormalTok{)}
\NormalTok{  fname_cs =}\StringTok{ }\KeywordTok{paste0}\NormalTok{(cs_name, }\StringTok{".RDs"}\NormalTok{)}
  
\NormalTok{  sa_name =}\StringTok{ }\KeywordTok{paste}\NormalTok{(}\StringTok{"SA"}\NormalTok{, y_var, }\DataTypeTok{sep =} \StringTok{"_"}\NormalTok{)}
\NormalTok{  fname_sa =}\StringTok{ }\KeywordTok{paste0}\NormalTok{(sa_name, }\StringTok{".RDs"}\NormalTok{)}
  
  \KeywordTok{load}\NormalTok{(}\KeywordTok{file.path}\NormalTok{(}\KeywordTok{here}\NormalTok{(), }\StringTok{"output"}\NormalTok{, }\StringTok{"models"}\NormalTok{, fname_twfe))}
  \KeywordTok{load}\NormalTok{(}\KeywordTok{file.path}\NormalTok{(}\KeywordTok{here}\NormalTok{(), }\StringTok{"output"}\NormalTok{, }\StringTok{"models"}\NormalTok{, fname_cs))}
  \KeywordTok{load}\NormalTok{(}\KeywordTok{file.path}\NormalTok{(}\KeywordTok{here}\NormalTok{(), }\StringTok{"output"}\NormalTok{, }\StringTok{"models"}\NormalTok{, fname_sa))}
  \KeywordTok{load}\NormalTok{(}\KeywordTok{file.path}\NormalTok{(}\KeywordTok{here}\NormalTok{(), }\StringTok{"output"}\NormalTok{, }\StringTok{"models"}\NormalTok{, fname_bacon))}
  
\NormalTok{  estimates =}\StringTok{ }\KeywordTok{c}\NormalTok{(y_var, model_fe[[}\StringTok{"coefficients"}\NormalTok{]][[}\StringTok{"shalll"}\NormalTok{]], cs_model[[}\StringTok{"overall.att"}\NormalTok{]])}
\NormalTok{  se =}\StringTok{ }\KeywordTok{c}\NormalTok{(y_var, model_fe[[}\StringTok{"se"}\NormalTok{]][[}\StringTok{"shalll"}\NormalTok{]], cs_model[[}\StringTok{"overall.se"}\NormalTok{]])}
\NormalTok{  p =}\StringTok{ }\KeywordTok{c}\NormalTok{(y_var, model_fe[[}\StringTok{"coeftable"}\NormalTok{]][[}\StringTok{"Pr(>|t|)"}\NormalTok{]][}\DecValTok{1}\NormalTok{], cs_model[[}\StringTok{"overall.se"}\NormalTok{]])}
  
  \KeywordTok{rbind}\NormalTok{(estimates, se)}
\NormalTok{\} }\OperatorTok\StringTok{ }
\StringTok{  }\KeywordTok{as.data.frame}\NormalTok{()}

\KeywordTok{colnames}\NormalTok{(effects) =}\StringTok{ }\KeywordTok{c}\NormalTok{(}\StringTok{"Target Variable"}\NormalTok{, }\StringTok{"TWFE"}\NormalTok{, }\StringTok{"Calloway_SantAnna"}\NormalTok{)}
\KeywordTok{rownames}\NormalTok{(effects) =}\StringTok{ }\OtherTok{NULL}

\NormalTok{effects}\OperatorTok{$}\NormalTok{TWFE =}\StringTok{ }\KeywordTok{round}\NormalTok{(}\KeywordTok{as.numeric}\NormalTok{(effects}\OperatorTok{$}\NormalTok{TWFE), }\DecValTok{4}\NormalTok{)}
\NormalTok{effects}\OperatorTok{$}\NormalTok{Calloway_SantAnna =}\StringTok{ }\KeywordTok{round}\NormalTok{(}\KeywordTok{as.numeric}\NormalTok{(effects}\OperatorTok{$}\NormalTok{Calloway_SantAnna), }\DecValTok{4}\NormalTok{)}

\NormalTok{effects =}\StringTok{ }\NormalTok{effects }\OperatorTok\StringTok{ }
\StringTok{  }\KeywordTok{mutate}\NormalTok{(}\StringTok{"Target Variable"}\NormalTok{ =}\StringTok{ }\KeywordTok{ifelse}\NormalTok{(}\KeywordTok{row_number}\NormalTok{()}\OperatorTok\DecValTok{2} \OperatorTok{==}\StringTok{ }\DecValTok{0}\NormalTok{, }\StringTok{""}\NormalTok{, y_vars)) }\OperatorTok\StringTok{ }
\StringTok{  }\KeywordTok{mutate}\NormalTok{(}\DataTypeTok{TWFE =} \KeywordTok{ifelse}\NormalTok{(}\KeywordTok{row_number}\NormalTok{()}\OperatorTok\DecValTok{2} \OperatorTok{==}\StringTok{ }\DecValTok{0}\NormalTok{, }\KeywordTok{paste0}\NormalTok{(}\StringTok{"("}\NormalTok{, TWFE, }\StringTok{")"}\NormalTok{), TWFE), }
         \DataTypeTok{Calloway_SantAnna =} \KeywordTok{ifelse}\NormalTok{(}\KeywordTok{row_number}\NormalTok{()}\OperatorTok\DecValTok{2} \OperatorTok{==}\StringTok{ }\DecValTok{0}\NormalTok{, }\KeywordTok{paste0}\NormalTok{(}\StringTok{"("}\NormalTok{,Calloway_SantAnna, }\StringTok{")"}\NormalTok{),Calloway_SantAnna))}

\NormalTok{effects_table =}\StringTok{ }\KeywordTok{stargazer}\NormalTok{(effects,}
                          \DataTypeTok{type=} \StringTok{"text"}\NormalTok{,}
                          \DataTypeTok{summary =} \OtherTok{FALSE}\NormalTok{,}
                          \DataTypeTok{rownames =} \OtherTok{FALSE}\NormalTok{)}
\end{Highlighting}
\end{Shaded}

\begin{verbatim}
## 
## ==========================================
## Target Variable   TWFE   Calloway_SantAnna
## ------------------------------------------
## lvio            -0.0572        -0.01      
##                 (0.0234)     (0.0248)     
## lmur             0.0085       0.0129      
##                 (0.0137)     (0.0122)     
## laga            -0.0504       -0.0486     
##                 (0.0396)     (0.0271)     
## lbur            -0.0494        0.02       
##                 (0.028)      (0.0298)     
## laut            -0.0509       0.0048      
##                 (0.0305)     (0.0419)     
## lpro            -0.0178       0.0388      
##                 (0.0313)      (0.037)     
## lrap            -0.0243       -0.0158     
##                 (0.0194)     (0.0157)     
## lrob             0.014        0.0303      
##                 (0.013)      (0.0159)     
## llar             0.0345       0.0107      
##                 (0.0297)     (0.0428)     
## ------------------------------------------
\end{verbatim}

\begin{Shaded}
\begin{Highlighting}[]
\KeywordTok{write}\NormalTok{(effects_table, }\DataTypeTok{file=}\KeywordTok{file.path}\NormalTok{(}\KeywordTok{here}\NormalTok{(), }\StringTok{"output"}\NormalTok{, }\StringTok{"tables"}\NormalTok{, }\StringTok{"effects.txt"}\NormalTok{))}
\end{Highlighting}
\end{Shaded}

\end{document}
